\documentclass[11pt,a4paper]{article}

\usepackage[utf8]{inputenc}

\textheight 24cm
\textwidth 16cm
\oddsidemargin 0cm
\topmargin -0.5cm

\title{\textsc{FLIHABI} - File format description}
\author{The \textsc{FLIHABI} team}

\begin{document}

\maketitle
\newpage

\section{The Tolk virtual machine file fromat}

In the \textsc{Flihabi} projet, the virtual machine (VM), i.e. \texttt{Tolk},
a file format is provided as an alternative to a common \texttt{elf} file. This
document aims to describe this format, in a formal way. In case of doubt or
misunderstanding, please concider contacting the \textsc{Flihabi} team.

\subsection{File header}

A tolk-file begins with a header composed of some fields used to serialize the
compiled bytecode's data needed by the VM in order to run the program. This
header is as follows:

\begin{verbatim}
 - A magic number composed of the values 0x54, 0x4F, 0x4C and 0x4B. (4*4 bytes)
 - The entry point of the program (pointer relative to the begining of the file)
   (4 bytes)
\end{verbatim}

\subsection{Symbol table}

The symbol table stores the symboles used in the program (a symbol is just an
offset in the file). The symbol table can be seen as an associative array, which
maps the symbol's ID (offset in the file) to the symbol itself. The datas are
serialized as follows:

\begin{verbatim}
  - The symbol id's offset in the file. (4 bytes)
  - The symbol's offset in the file. (4 bytes)
  - ...
\end{verbatim}

\subsection{String table}

The string table is used to store the strings used in the program. A string
is composed of an ID, and the literal string itself. As for the symbol table,
it can be seen as a map, associating the string id's offset to the literal
string.

\begin{verbatim}
  - The string id's offset in the file (4 bytes)
  - The size of the string (4 bytes)
  - The litteral string (n bytes where n is the size of the string)
  - ...
\end{verbatim}

\subsection{Function table}

A function is represented as an offset in the bytecode (entry point of the
function), it's registers and the offset of the registers. As for the symtable
and the strtable, the functable can be seen as a map, associating the function's
id to it's data structure.

\begin{verbatim}
  - The function id's offset in the file (4 bytes)
  - The offset of the function in the bytecode (4 bytes)
  - The function's registers (4 bytes)
  - The function registers' offset (8 bytes)
  - ...
\end{verbatim}

\subsection{Bytecode}

The bytecode of the program, generated by the Farango compiler. This bytecode
can be as long as needed, and stops at the end of the file.

\end{document}
